\documentclass[aspectratio=169]{beamer}
\usetheme{metropolis}

\title{Example Document to Recreate with \texttt{beamer} in \LaTeX}
\subtitle{Week \textbf{3} \textit{Exercise}}

\author{\href{https://ihnwhiheo.github.io/}{Ihnwhi Heo} \\ \texttt{\href{mailto:i.heo@uu.nl}{i.heo@uu.nl}} \\}

\date{Fall 2020 \\ Markup Languages and Reproducible Programming in Statistics}

\institute{
  Department of Methodology and Statistics\\
  Faculty of Social and Behavioural Sciences\\
  Utrecht University, The Netherlands\\[1ex]
}

\begin{document}

\maketitle

\begin{frame}{Outline}

\tableofcontents

\end{frame}

\section{Working with equations}

\begin{frame}{Working with equations}
We define a set of equations as
\begin{equation}
a = b + c^2,
\end{equation}
\begin{equation}
a-c^2 = b,
\end{equation}
\begin{equation}
\text{left side} = \text{right side},
\end{equation}
\begin{equation}
\text{left side} + \text{something} \geq \text{right side},
\end{equation}
for all something $>$ 0.
\end{frame}

\subsection{Aligning the same equations}

\subsection{Aligning the same equations}
\begin{frame}
\frametitle{Aligning the same equations}
Aligning the equations by the equal sign gives a much better view into the placements of the separate equation components. 
\begin{align}
a&=b+c^2,\\
a-c^2&=b,\\
\text{left side} &= \text{right side},\\
\text{left side} + \text{something} &\geq \text{right side},
\end{align}
\end{frame}

\subsection{Omit equation numbering}

\begin{frame}{Omit equation numbering}
Alternatively, the equation numbering can be omitted.
\begin{align*}
a&=b+c^2\\
a-c^2&=b\\
\text{left side} &= \text{right side}\\
\text{left side} + \text{something} &\geq \text{right side}
\end{align*}
\end{frame}

\subsection{Ugly alignment}

\begin{frame}{Ugly alignment}
Some components do not look well, when aligned. Especially equations with different heights and spacing. For example,
\begin{align}
E &= mc^2,\\
m &= \frac{E}{c^2},\\
c &= \sqrt{ \frac{E}{m} }.
\end{align}
Take that into account.
\end{frame}

\section{Discussion}

\begin{frame}{Discussion}

So far, we have discussed the benefits of Bayesian inference in performing statistical modeling. From now on, let's discuss the following points:

\begin{itemize}

\item Advantages and disadvantages of the Bayesian approach compared to the frequentist approach
\item Bayesian hypothesis testing with BF and PMP
\item Setting prior distributions sensibly as well as wisely
\item Practice Bayesian statistics with JASP (Jeffrey's Amazing Statistics Program)

\end{itemize}

\end{frame}

\end{document}